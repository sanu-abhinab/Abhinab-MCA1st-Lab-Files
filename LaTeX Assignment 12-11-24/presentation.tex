\documentclass{beamer}
\usepackage{amsmath}
\usepackage{color}

\title{Jump Statements in C Programming}
\author{Abhinab Dowerah, Roll No. 10, MCA 1st}
\date{\today}

\begin{document}

\begin{frame}
    \titlepage
\end{frame}

\begin{frame}{Introduction}
    \begin{itemize}
        \item Jump statements in C allow for the control of program flow by skipping over parts of code.
        \item They provide an alternative to structured control statements like loops and conditionals.
        \item There are four types of Jump Statements:
        \begin{itemize}
            \item \texttt{break}
            \item \texttt{continue}
            \item \texttt{goto}
            \item \texttt{return}
        \end{itemize}
    \end{itemize}
\end{frame}

\begin{frame}{The \texttt{break} Statement}
    \begin{itemize}
        \item The break statement exits or terminate the loop or switch statement based on a certain condition, without executing the remaining code.
        \item The statements inside the loop are executed sequentially.
    \end{itemize}
    \vspace{0.3cm}
    \textbf{Example: Exit Loop on a Specific Condition}
    \begin{block}{}
        \texttt{\#include <stdio.h>}\\
        \texttt{int main() \{}\\
        \texttt{\ \ for (int i = 1; i <= 10; i++) \{}\\
        \texttt{\ \ \ \ if (i == 5) \{ break; \}}\\
        \texttt{\ \ \ \ printf("\%d ", i);}\\
        \texttt{\ \ \}}\\
        \texttt{\ \ return 0;}\\
        \texttt{\}}
    \end{block}
    \textbf{Real-life Example:} Exiting a search once an item is found in a list.
\end{frame}

\begin{frame}{The \texttt{continue} Statement}
    \begin{itemize}
        \item The continue statement in C is used to skip the remaining code after the continue statement within a loop and jump to the next iteration of the loop.
        
    \end{itemize}
    \vspace{0.3cm}
    \textbf{Example: Skip Printing Even Numbers}
    \begin{block}{}
        \texttt{\#include <stdio.h>}\\
        \texttt{int main() \{}\\
        \texttt{\ \ for (int i = 1; i <= 10; i++) \{}\\
        \texttt{\ \ \ \ if (i \% 2 == 0) \{ continue; \}}\\
        \texttt{\ \ \ \ printf("\%d ", i);}\\
        \texttt{\ \ \}}\\
        \texttt{\ \ return 0;}\\
        \texttt{\}}
    \end{block}
    \textbf{Real-life Example:} Skipping over irrelevant data points in data analysis.
\end{frame}

\begin{frame}{The \texttt{goto} Statement}
    \begin{itemize}
        \item The goto statement is used to jump to a specific point from anywhere in a function.
        \item It is used to transfer the program control to a labeled statement within the same function.
    \end{itemize}
    \vspace{0.3cm}
    \textbf{Example: Using \texttt{goto} for Error Handling}
    \begin{block}{}
        \texttt{\#include <stdio.h>}\\
        \texttt{int main() \{}\\
        \texttt{\ \ int i = 10;}\\
        \texttt{\ \ if (i < 0) goto error;}\\
        \texttt{\ \ printf("Processing...\textbackslash n");}\\
        \texttt{\ \ error: printf("Error encountered!\textbackslash n");}\\
        \texttt{\ \ return 0;}\\
        \texttt{\}}
    \end{block}
    \textbf{Real-life Example:} Handling unexpected errors in low-level system programming.
\end{frame}

\begin{frame}{The \texttt{return} Statement}
    \begin{itemize}
        \item The return statement in C is used to terminate the execution of a function and return a value to the caller.
        \item It is commonly used to provide a result back to the calling code.
    \end{itemize}
    \vspace{0.3cm}
    \textbf{Example: Return with a Value}
    \begin{block}{}
        \texttt{\#include <stdio.h>}\\
        \texttt{int add(int a, int b) \{}\\
        \texttt{\ \ return a + b;}\\
        \texttt{\}}\\
        \texttt{int main() \{}\\
        \texttt{\ \ int sum = add(3, 4);}\\
        \texttt{\ \ printf("Sum: \%d", sum);}\\
        \texttt{\ \ return 0;}\\
        \texttt{\}}
    \end{block}
    \textbf{Real-life Example:} Returning values from helper functions in larger programs.
\end{frame}

\begin{frame}{Conclusion}
    \begin{itemize}
        \item Jump statements provide flexibility in managing program flow.
        \item While powerful, their misuse can lead to unreadable code.
        \item Use them judiciously to improve code efficiency and clarity.
    \end{itemize}
\end{frame}

\end{document}